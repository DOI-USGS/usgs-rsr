% Options for packages loaded elsewhere
\PassOptionsToPackage{unicode$for(hyperrefoptions)$,$hyperrefoptions$$endfor$}{hyperref}
\PassOptionsToPackage{hyphens}{url}
$if(colorlinks)$
\PassOptionsToPackage{dvipsnames,svgnames*,x11names*}{xcolor}
$endif$
$if(dir)$
$if(latex-dir-rtl)$
\PassOptionsToPackage{RTLdocument}{bidi}
$endif$
$endif$
$if(CJKmainfont)$
\PassOptionsToPackage{space}{xeCJK}
$endif$


\documentclass[12pt]{article}

\usepackage[letterpaper, margin = 0.75in]{geometry}
\usepackage[hyphens]{url}
\usepackage{multicol}
\usepackage{mathptmx}
\usepackage{changepage}
\usepackage{sectsty, textcomp}
\usepackage{xcolor}
\usepackage{fancyhdr}
\usepackage{lastpage}
\usepackage[colorlinks]{hyperref}
\usepackage{titlesec}
\usepackage{calc}
\usepackage{amssymb,amsmath}
\usepackage{ifxetex,ifluatex}
\usepackage{etaremune}
\usepackage{enumitem}


\ifnum 0\ifxetex 1\fi\ifluatex 1\fi=0 % if pdftex
  \usepackage[$if(fontenc)$$fontenc$$else$T1$endif$]{fontenc}
  \usepackage[utf8]{inputenc}
  \usepackage{textcomp} % provide euro and other symbols
\else % if luatex or xetex
$if(mathspec)$
  \ifxetex
    \usepackage{mathspec}
  \else
    \usepackage{unicode-math}
  \fi
$else$
  \usepackage{unicode-math}
$endif$
  \defaultfontfeatures{Scale=MatchLowercase}
  \defaultfontfeatures[\rmfamily]{Ligatures=TeX,Scale=1}
$if(mainfont)$
  \setmainfont[$for(mainfontoptions)$$mainfontoptions$$sep$,$endfor$]{$mainfont$}
$endif$
$if(sansfont)$
  \setsansfont[$for(sansfontoptions)$$sansfontoptions$$sep$,$endfor$]{$sansfont$}
$endif$
$if(monofont)$
  \setmonofont[$for(monofontoptions)$$monofontoptions$$sep$,$endfor$]{$monofont$}
$endif$
$for(fontfamilies)$
  \newfontfamily{$fontfamilies.name$}[$for(fontfamilies.options)$$fontfamilies.options$$sep$,$endfor$]{$fontfamilies.font$}
$endfor$
$if(mathfont)$
$if(mathspec)$
  \ifxetex
    \setmathfont(Digits,Latin,Greek)[$for(mathfontoptions)$$mathfontoptions$$sep$,$endfor$]{$mathfont$}
  \else
    \setmathfont[$for(mathfontoptions)$$mathfontoptions$$sep$,$endfor$]{$mathfont$}
  \fi
$else$
  \setmathfont[$for(mathfontoptions)$$mathfontoptions$$sep$,$endfor$]{$mathfont$}
$endif$
$endif$
$if(CJKmainfont)$
  \ifxetex
    \usepackage{xeCJK}
    \setCJKmainfont[$for(CJKoptions)$$CJKoptions$$sep$,$endfor$]{$CJKmainfont$}
  \fi
$endif$
$if(luatexjapresetoptions)$
  \ifluatex
    \usepackage[$for(luatexjapresetoptions)$$luatexjapresetoptions$$sep$,$endfor$]{luatexja-preset}
  \fi
$endif$
$if(CJKmainfont)$
  \ifluatex
    \usepackage[$for(luatexjafontspecoptions)$$luatexjafontspecoptions$$sep$,$endfor$]{luatexja-fontspec}
    \setmainjfont[$for(CJKoptions)$$CJKoptions$$sep$,$endfor$]{$CJKmainfont$}
  \fi
$endif$
\fi
$if(beamer)$
$if(theme)$
\usetheme[$for(themeoptions)$$themeoptions$$sep$,$endfor$]{$theme$}
$endif$
$if(colortheme)$
\usecolortheme{$colortheme$}
$endif$
$if(fonttheme)$
\usefonttheme{$fonttheme$}
$endif$
$if(mainfont)$
\usefonttheme{serif} % use mainfont rather than sansfont for slide text
$endif$
$if(innertheme)$
\useinnertheme{$innertheme$}
$endif$
$if(outertheme)$
\useoutertheme{$outertheme$}
$endif$
$endif$
% Use upquote if available, for straight quotes in verbatim environments
\IfFileExists{upquote.sty}{\usepackage{upquote}}{}
\IfFileExists{microtype.sty}{% use microtype if available
  \usepackage[$for(microtypeoptions)$$microtypeoptions$$sep$,$endfor$]{microtype}
  \UseMicrotypeSet[protrusion]{basicmath} % disable protrusion for tt fonts
}{}
$if(indent)$
$else$
\makeatletter
\@ifundefined{KOMAClassName}{% if non-KOMA class
  \IfFileExists{parskip.sty}{%
    \usepackage{parskip}
  }{% else
    \setlength{\parindent}{0pt}
    \setlength{\parskip}{6pt plus 2pt minus 1pt}}
}{% if KOMA class
  \KOMAoptions{parskip=half}}
\makeatother
$endif$
$if(verbatim-in-note)$
\usepackage{fancyvrb}
$endif$
\usepackage{xcolor}
\IfFileExists{xurl.sty}{\usepackage{xurl}}{} % add URL line breaks if available
\IfFileExists{bookmark.sty}{\usepackage{bookmark}}{\usepackage{hyperref}}

\hypersetup{colorlinks=true,linkbordercolor=red,linkcolor=blue, urlcolor=blue}


\urlstyle{same} % disable monospaced font for URLs
$if(verbatim-in-note)$
\VerbatimFootnotes % allow verbatim text in footnotes
$endif$
$if(geometry)$
$if(beamer)$
\geometry{$for(geometry)$$geometry$$sep$,$endfor$}
$else$
\usepackage[$for(geometry)$$geometry$$sep$,$endfor$]{geometry}
$endif$
$endif$
$if(beamer)$
\newif\ifbibliography
$endif$
$if(listings)$
\usepackage{listings}
\newcommand{\passthrough}[1]{#1}
\lstset{defaultdialect=[5.3]Lua}
\lstset{defaultdialect=[x86masm]Assembler}
$endif$
$if(lhs)$
\lstnewenvironment{code}{\lstset{language=Haskell,basicstyle=\small\ttfamily}}{}
$endif$
$if(highlighting-macros)$
$highlighting-macros$
$endif$
$if(tables)$
\usepackage{longtable,booktabs}
$if(beamer)$
\usepackage{caption}
% Make caption package work with longtable
\makeatletter
\def\fnum@table{\tablename~\thetable}
\makeatother
$else$
% Correct order of tables after \paragraph or \subparagraph
\usepackage{etoolbox}
\makeatletter
\patchcmd\longtable{\par}{\if@noskipsec\mbox{}\fi\par}{}{}
\makeatother
% Allow footnotes in longtable head/foot
\IfFileExists{footnotehyper.sty}{\usepackage{footnotehyper}}{\usepackage{footnote}}
\makesavenoteenv{longtable}
$endif$
$endif$
$if(graphics)$
\usepackage{graphicx}
\makeatletter
\def\maxwidth{\ifdim\Gin@nat@width>\linewidth\linewidth\else\Gin@nat@width\fi}
\def\maxheight{\ifdim\Gin@nat@height>\textheight\textheight\else\Gin@nat@height\fi}
\makeatother
% Scale images if necessary, so that they will not overflow the page
% margins by default, and it is still possible to overwrite the defaults
% using explicit options in \includegraphics[width, height, ...]{}
\setkeys{Gin}{width=\maxwidth,height=\maxheight,keepaspectratio}
% Set default figure placement to htbp
\makeatletter
\def\fps@figure{htbp}
\makeatother
$endif$
$if(links-as-notes)$
% Make links footnotes instead of hotlinks:
\DeclareRobustCommand{\href}[2]{#2\footnote{\url{#1}}}
$endif$
$if(strikeout)$
\usepackage[normalem]{ulem}
% Avoid problems with \sout in headers with hyperref
\pdfstringdefDisableCommands{\renewcommand{\sout}{}}
$endif$
\setlength{\emergencystretch}{3em} % prevent overfull lines
\providecommand{\tightlist}{%
  \setlength{\itemsep}{0pt}\setlength{\parskip}{0pt}}
$if(numbersections)$
\setcounter{secnumdepth}{$if(secnumdepth)$$secnumdepth$$else$5$endif$}
$else$
\setcounter{secnumdepth}{-\maxdimen} % remove section numbering
$endif$
$if(beamer)$
$else$
$if(block-headings)$
% Make \paragraph and \subparagraph free-standing
\ifx\paragraph\undefined\else
  \let\oldparagraph\paragraph
  \renewcommand{\paragraph}[1]{\oldparagraph{#1}\mbox{}}
\fi
\ifx\subparagraph\undefined\else
  \let\oldsubparagraph\subparagraph
  \renewcommand{\subparagraph}[1]{\oldsubparagraph{#1}\mbox{}}
\fi
$endif$
$endif$

\pagestyle{fancy}
\renewcommand{\headrulewidth}{0pt}
\fancyhead{} 
 \cfoot{\thepage\ of \pageref*{LastPage}}
  \fancypagestyle{plain}{%
  \renewcommand{\headrulewidth}{0pt}%
  \fancyhf{}%
  \fancyfoot[C]{\footnotesize \thepage\ of \pageref*{LastPage}}%
}


$for(header-includes)$
$header-includes$
$endfor$
$if(lang)$
\ifxetex
  % Load polyglossia as late as possible: uses bidi with RTL langages (e.g. Hebrew, Arabic)
  \usepackage{polyglossia}
  \setmainlanguage[$polyglossia-lang.options$]{$polyglossia-lang.name$}
$for(polyglossia-otherlangs)$
  \setotherlanguage[$polyglossia-otherlangs.options$]{$polyglossia-otherlangs.name$}
$endfor$
\else
  \usepackage[shorthands=off,$for(babel-otherlangs)$$babel-otherlangs$,$endfor$main=$babel-lang$]{babel}
$if(babel-newcommands)$
  $babel-newcommands$
$endif$
\fi
$endif$
$if(dir)$
\ifxetex
  % Load bidi as late as possible as it modifies e.g. graphicx
  \usepackage{bidi}
\fi
\ifnum 0\ifxetex 1\fi\ifluatex 1\fi=0 % if pdftex
  \TeXXeTstate=1
  \newcommand{\RL}[1]{\beginR #1\endR}
  \newcommand{\LR}[1]{\beginL #1\endL}
  \newenvironment{RTL}{\beginR}{\endR}
  \newenvironment{LTR}{\beginL}{\endL}
\fi
$endif$
$if(natbib)$
\usepackage[$natbiboptions$]{natbib}
\bibliographystyle{$if(biblio-style)$$biblio-style$$else$plainnat$endif$}
$endif$
$if(biblatex)$
\usepackage[$if(biblio-style)$style=$biblio-style$,$endif$$for(biblatexoptions)$$biblatexoptions$$sep$,$endfor$]{biblatex}
$for(bibliography)$
\addbibresource{$bibliography$}
$endfor$
$endif$
$if(csl-refs)$
\newlength{\cslhangindent}
\setlength{\cslhangindent}{1.5em}
\newenvironment{cslreferences}%
  {$if(csl-hanging-indent)$\setlength{\parindent}{0pt}%
  \everypar{\setlength{\hangindent}{\cslhangindent}}\ignorespaces$endif$}%
  {\par}
$endif$


\titleformat{\section} {\centering\large\bfseries}{\makebox[1.0em][l]{\bfseries\thesection}}{1em}{}
\titleformat{\subsection} {\large\itshape}{\makebox[1.0em][l]{\thesubsection}}{1em}{}%

\addtolength{\leftskip}{\widthof{\normalfont\bfseries\makebox[2.0em]{}}}% Indent text

\titlespacing\section{0pt}{6pt plus 4pt minus 2pt}{6pt plus 2pt minus 2pt}
% spacing: how to read {12pt plus 4pt minus 2pt}
%           12pt is what we would like the spacing to be
%           plus 4pt means that TeX can stretch it by at most 4pt
%           minus 2pt means that TeX can shrink it by at most 2pt
%       This is one example of the concept of, 'glue', in TeX

\renewcommand\thesubsection{\Alph{subsection}.}

\renewcommand\thesubsubsection{\Alph{subsection}.\arabic{subsubsection}}


\setlist[itemize]{topsep=0pt,itemsep=1ex,partopsep=1ex,parsep=1ex}
\setenumerate[1]{topsep=0pt,itemsep=1ex,partopsep=1ex,parsep=1ex, label=\thesubsection\arabic*}
\setenumerate[2]{topsep=0pt,itemsep=1ex,partopsep=1ex,parsep=1ex, leftmargin=*, label=\theenumi.DR}


$if(title)$
\title{\large{U.S.~Geological Survey -- $type$}}
\date{\vspace{-10ex}}
$endif$

\begin{document}
$if(has-frontmatter)$
\frontmatter
$endif$
$if(title)$
$if(beamer)$
\frame{\titlepage}
$else$
\maketitle
$endif$
$if(abstract)$
\begin{abstract}
$abstract$
\end{abstract}
$endif$
$endif$


$for(include-before)$
$include-before$

$endfor$

\sectionfont{\bfseries\normalsize\raggedright}

\setlength{\parindent}{0em}

\begin{center}
\section*{$scientist$ $pronouns$\\ 
  $title$, $seriesAndGrade$}


  \begin{tabular}{r l}
    \textrm{Center name}: & \textrm{$centerName$} \nonumber \\
    \textrm{Duty station}: & \textrm{$dutyStation$} \nonumber \\
    \textrm{Date of entrance on duty to federal service}: &
                                                            \textrm{$entrance$}
                                                            \nonumber \\
    \textrm{Date of conversion to research scientist}: &
                                                         \textrm{$conversion$}
                                                         \nonumber \\
    \textrm{Date of last promotion}: & \textrm{$promotion$} \nonumber \\
    \textrm{Review Cycle}: & \textrm{$cycle$} \nonumber
  \end{tabular} \\ \vspace{0.5cm}

\section*{Research Interests and Expertise}
\vspace{-0.2cm}
{\color{red} \footnotesize{up to 5 keywords or brief phrases}}\\
$keyword1$
$keyword2$
$keyword3$
$keyword4$
$keyword5$
\end{center}


$if(has-frontmatter)$


\backmatter
$endif$
$if(natbib)$
$if(bibliography)$
$if(biblio-title)$
$if(has-chapters)$
\renewcommand\bibname{$biblio-title$}
$else$
\renewcommand\refname{$biblio-title$}
$endif$
$endif$
$if(beamer)$
\begin{frame}[allowframebreaks]{$biblio-title$}
  \bibliographytrue
$endif$
  \bibliography{$for(bibliography)$$bibliography$$sep$,$endfor$}
$if(beamer)$
\end{frame}
$endif$

$endif$
$endif$
$if(biblatex)$
$if(beamer)$
\begin{frame}[allowframebreaks]{$biblio-title$}
  \bibliographytrue
  \printbibliography[heading=none]
\end{frame}
$else$
\printbibliography$if(biblio-title)$[title=$biblio-title$]$endif$
$endif$

$endif$
$for(include-after)$
$include-after$

$endfor$

\newpage

\section*{Research Environment}

\vspace{-0.2cm}
\begin{center}
  {\color{red} \footnotesize{up to 2,500 characters, including spaces}}\\
\end{center}

This section is meant to be an abstract-like description of your position to provide context for panel members. \\

Provide a concise, summary-level, introductory statement describing your current position’s purpose and your research environment, including the scope of the research you conduct (for example, topical, geographic, taxonomic), your primary collaborators, and your major sources (but not amounts) of funding.
Include relevant information about your typical role on research teams (for example, principal investigator, team leader, or team member).\\

As applicable, document other major assignments or responsibilities, such as standing representation or committee assignments and any non-research duties. 
Include the percentage of time spent on these duties.
This may include, for example: 

\begin{itemize}
\item technical assistance related to an assignment, committee assignments, technology transfer, or consultation;
\item facility or laboratory management;
\item special assignments (for example, safety officer responsibilities); or
\item substantive supervisory duties.
\end{itemize}

This should describe the duties of your position, but not include outcomes or accomplishments.
Detailed lists of projects, examples of accomplishments, and other specifics are better included in the following sections.
Do not include information about your physical work environment. \\

This section is limited to 2,500 characters, including spaces (about three-quarters of a page).
If this section exceeds the character limit, your Scientist Record may be returned to you for editing, which may delay your grade evaluation.


\newpage

\section*{Factor 1: Research Assignment}

\vspace{-0.2cm}
\begin{center}
  {\color{red} \footnotesize{up to 7,000 characters, including spaces}}\\
\end{center}


%Describe your research assignment and define the primary areas(s) of work, with attention to:
%?	scope and breadth of the research;
%?	availability of related research studies or products, or protocols;
%?	complexity of assignment (e.g., unknowns; critical obstacles; difficulty)
%?	extent to which objectives can be defined;
%?	approach(s)taken to solve scientific or engineering problems;
%?	variety and depth of knowledge and expertise required to solve problems;
%?	extent and complexity of the required validation process;
%?	necessity to translate abstract concepts into easily understood statements of theory or models;
%?	importance of and utility of end products in solving problems and for use by managers, decision and policy makers; and
%?	importance of work in opening new areas of research or development.
%
%Do not describe each individual study or project but rather focus on the major themes of your research program. Individual study/project titles and funding sources can be listed in Appendix A and can be referenced in the narrative. 
%%2-page limit

%?	scope and breadth of the research;

This factor deals with the nature, scope, and characteristics of the researcher's current assignment.
Award a factor level that reflects the norm of current assignments,
rather than atypical projects.
Research assignments are directly dependent upon the individual
qualities of the researcher and the inherent difficulty of the
research problems.
Work commonly expands commensurate with the researcher's motivation,
capability, and creativity. \\

\paragraph*{\textbf{Projects and Teams -- }}  For project and team members, base the factor
level only on the specific projects or portion of projects for which
the researcher is responsible.
For project managers, base the factor level on the scope and character
of the total project.

\paragraph*{\textbf{Primary Considerations --}}
 In evaluating this factor consider the
following:
\begin{itemize}
  \item assignment scope and complexity, objectives, and means of accomplishment;
  \item problem breadth and depth; 
  \item availability of related research studies; 
  \item extent to which objectives can be defined; 
  \item number of unknowns and critical obstacles; 
  \item variety and depth of knowledge and expertise required to solve problems;
  \item extent and complexity of the required validation process; 
  \item necessity to translate abstract concepts into easily understood statements of theory or models, and to determine how best to disseminate information or transfer research findings;
  \item utility of the end product in solving the initial problem and in opening new areas of investigation; and
  \item expected impact of end results, products, or outcomes.
\end{itemize}

\subsubsection*{Factor 1 --  Level A (2 points)}

Research assignments have the following characteristics:
\begin{itemize}
  \item readily definable objectives; 
  \item limited in scope to investigating specific phenomena or problems, or are segments of related investigations; 
  \item require fairly conventional techniques; 
  \item involve applying existing theory or methods to areas previously investigated, but under different conditions, or involve adapting previous studies in light of changes in theory or improved techniques and instrumentation; and
  \item result in contributions that add to scientific and professional knowledge or support developing new or improved methods and techniques. 
\end{itemize}

\noindent The researcher typically works as a project or team member. 

\subsubsection*{Factor 1 --  Level C (6 points)}

Research assignments have the following characteristics:
\begin{itemize}
  \item the scope is broad and complex, requiring a series of comprehensive and conceptually related phases and studies; 
  \item problems are difficult to define;
  \item require sophisticated research techniques; and
  \item result in contributions that:
  \begin{itemize}
    \item answer important questions in the field;
    \item account for previously unexplained phenomena; 
    \item open significant new avenues for further study;
    \item confirm or modify a scientific theory or methodology;
    \item lead to important changes in existing products, methods, techniques, processes, or practices; or
    \item are definitive of a specific topic area. 
  \end{itemize}
\end{itemize}

\noindent The researcher typically works as a project member or as a primary investigator. 
\subsubsection*{Factor 1 --  Level E (10 points)}

Research assignments have the following characteristics:
\begin{itemize}
  \item  the scope and complexity are at a level requiring subdivision into separate phases, some of  which are considerably broad and complex; 
  \item problems are exceptionally difficult and unyielding to investigation; 
  \item require unconventional or novel approaches or complex research techniques; and 
  \item results may include:
  \begin{itemize}
    \item a major advance or opening of the way for extensive related development; 
    \item progress in areas of exceptional interest to the scientific and professional community; 
    \item important changes in theories, methods, and techniques; 
    \item opening significant new avenues for further study; or 
    \item contributions answering important questions in the field.
  \end{itemize}
\end{itemize} 

\noindent The researcher typically works as a primary investigator but may also be a project member.



\newpage

\section*{Factor 2: Supervisory Controls/Supervision Received}

\vspace{-0.2cm}
\begin{center}
  {\color{red} \footnotesize{up to 3,500 characters, including spaces}}\\
\end{center}


%This factor addresses your level of independence. Do you work on
%projects as assigned under close supervision, which is where most
%everyone starts, or do you define and pursue your own research
%projects  with little or no supervisory controls?
%
%Describe your relationship with your supervisor and your current level of independent performance.  Specify the following:
%?	what is your supervisor?s role (e.g., Science Center Director, supervisory scientist, ..);
%?	how work is assigned (e.g., consider nature of guidance provided, employee?s freedom in choosing a course of action);
%?	how independently work is performed;
%?	technical and administrative guidance and control exercised by supervisor
%?	how work is reviewed and the acceptance of your research results, conclusions and final products; and
%?	freedom to represent your research program to other organizations and agencies, policy makers and the public. 
%
%Use caution in distinguishing between consultation and supervisory control and guidance.
%1-page limit

This factor deals with the researcher’s current level of independent performance and the technical and administrative guidance and control the supervisor exercises over research work. 
Researchers may consult frequently with colleagues and collaborators.
Use caution in distinguishing between consultation and supervisory control and guidance

\textbf{Primary Considerations} -- In evaluating this factor, consider
the following:

\begin{itemize}
\item manner in which the supervisor assigns work;
\item researcher's freedom to determine a course of action; 
\item researcher's opportunity for procedural innovation; and 
\item degree of the supervisor's acceptance of the researcher's recommendations, decisions, and 
  final products.
\end{itemize}
Researchers working on complex team projects not divided into smaller components exercise 
independent performance when they:
\begin{itemize}
\item participate fully as a professionally responsible team member in substantive aspects of the 
  work; and 
\item make contributions equivalent to independently performing more
  limited research projects.
\end{itemize}

\subsubsection*{Factor 2 – Level A (2 points)}

The supervisor typically assigns specific problems along with general instructions on the scope and objectives of the study.
The supervisor or higher management makes any decisions to discontinue work, change emphasis, or change the research plan.
The researcher may suggest  studies and undertake them after receiving supervisory approval.
The supervisor reviews work for adequacy of method, completeness, and appropriate interpretation of results. \\

The researcher confers with the supervisor regarding problem definition, the relationship of the problem to the organization’s broader research goals, and developing a research plan. 
Supervisory or managerial direction and guidance help the researcher in the critical problem definition and planning stages, but do not negate the researcher’s responsibility for adequately completing these steps.  \\

The researcher is expected to:

\begin{itemize}
  \item assume responsibility for the study and pursue it to completion;
  \item solve problems ordinarily encountered in accomplishing the work with only occasional supervisory input;
  \item interpret results; and
  \item prepare entire, or sections of, reports and papers.
\end{itemize}

\subsubsection*{Factor 2 -- Level C (6 points)}

The supervisor may either assign a broad problem area to the researcher or allow the researcher to work with substantial freedom within an area of primary interest.
The researcher has substantial freedom to identify, define, and select specific projects, and to determine the most promising research strategies and problem approaches. \\

The supervisor:
\begin{itemize}
  \item approves plans calling for considerable investments of time or resources;
  \item makes final decisions concerning the direction of work and changes in or discontinuance of projects involving substantial research investments;
  \item relies on the researcher's professional judgment to such an extent that the researcher's recommendations are ordinarily followed; and
  \item  reviews final work and reports, principally to evaluate overall results, recommendations, and conclusions.
\end{itemize}

The researcher is responsible, with little technical direction, for: 
\begin{itemize}
  \item formulating hypotheses;
  \item developing and carrying out the research plan;
  \item determining equipment and other resource needs;
  \item keeping the supervisor informed of general plans and progress;
  \item addressing novel and difficult problems requiring modification of standard methods;
  \item analyzing and interpreting results;
  \item preparing comprehensive reports of findings; and
  \item working with users to interpret and implement research findings or technologies,
\end{itemize}


\subsubsection*{Factor 2 -- Level E (10 points)}

The supervisor provides broad administrative supervision, which is
generally limited to approving staffing, funds, and facilities, and
to providing broad guidance on agency policies and mandates.
Technical supervision is consultative in nature.
Management accepts the researcher's findings as technically
authoritative, as a basis for decisions, and as acceptable for review
by the scientific community.
The researcher, working within the framework of management objectives
and priorities, is responsible for:
\begin{itemize}
  \item formulating research plans and hypotheses; 
  \item carrying out the project plan; 
  \item interpreting findings and assessing their organizational and professional applicability; and 
  \item locating and exploring the most promising areas of research in relation to agency program needs and the state of the science or discipline.
\end{itemize}

\newpage

\section*{Factor 3: Guidelines and Originality}

\vspace{-0.2cm}
\begin{center}
  {\color{red} \footnotesize{up to 7,000 characters, including spaces}}\\
\end{center}

%This factor speaks to the originality of the research you are engaged
%in. Are you applying well known techniques and procedures, or
%adapting old methods to new problems, or creating an entirely new
%protocol to attack a research problem?

%
%Describe the creative thinking, analysis, synthesis, evaluation,
%judgment, resourcefulness and insight needed to perform your
%work. Identify existing scientific guides (or knowledge) and indicate
%the degree of applicability (e.g., how available/useful are technical
%handbooks, periodicals, reports, patent disclosures, guidance from
%technical specialists). Also describe the intrinsic difficulty in
%applying available guides to your current assignment?

%Address:
%?	original and independent creation of: research subjects, analysis, reasoning, evaluation, and judgment; and 
%?	originality in interpreting findings and translating findings into a form usable by others
%Describe originality (e.g., technical judgment, creativity, resourcefulness) needed to fill in, adapt, or extend theories, methods, and techniques.
%
%If you use established techniques for your field describe how you are applying those techniques to complex and original questions/problems.
%
%Reference information contained in appendices as appropriate.  
%
%2 page limit

This factor deals with the creative thinking, analysis, synthesis,
evaluation, judgment, resourcefulness, and insight characterizing the
work currently performed.\\

Guidelines usually consist of literature in the field, procedures,
instructions, or precedents and may be adapted or modified to meet the
requirements of the current assignment.
Features to be considered are:
\begin{itemize}
\item the extent and nature of available written guides; 
\item intrinsic difficulty encountered in applying guides in terms of
  their ready adaptability to the current assignment; and
\item the degree of judgment required in selecting, interpreting, and
  adapting guidelines.
\end{itemize}
In assessing the impact of creativity in the position, consider the requirement for: 
\begin{itemize}
\item original and independent creation, analysis, reasoning,
  evaluation, and judgment; and
\item originality in interpreting findings and translating findings
  into a form usable by others.
\end{itemize}

\subsubsection*{Factor 3 -- Degree Level A (2 points)}

Guideline include:
\begin{itemize}
  \item existing theories and methods generally applicable to the research problem; or
  \item materials that may contain some inconsistencies, be partially defined, or provide several possible approaches to the problem. 
\end{itemize}

Originality is demonstrated by:
\begin{itemize}
  \item developing a complete and adequate research design by selecting and adapting the most appropriate approach, methods, or techniques for the problem at hand; and
  \item limited extension or modification of procedures or techniques, as required.
\end{itemize}

\subsubsection*{Factor 3 -- Degree Level C (6 points)}

Guidelines:
\begin{itemize}
  \item consist of existing literature in the field of limited usefulness due to contradictions, critical 
  gaps, or limited applicability; or
  \item are largely absent because of the novel nature of the work. 
\end{itemize}

Originality is demonstrated by:
\begin{itemize}
  \item defining elusive or highly complex problems;
  \item developing productive hypotheses for testing;
  \item developing important new approaches, methods, and techniques;
  \item interpreting and relating significant results to other research findings; 
  \item developing and applying new techniques and original methods of attack to solve important problems presenting unprecedented or novel aspects;
  \item isolating and defining critical problem features; and
  \item adapting, extending, and synthesizing theory, principles, and techniques into original or innovative combinations or configurations. 
\end{itemize}

\subsubsection*{Factor 3 -- Degree Level E (10 points)}

Guidelines are almost nonexistent in pertinent literature.\\
Originality and creativity are demonstrated by:

\begin{itemize}
\item discovering complex theory or methodology; 
\item contributing significantly to the development of new theory or methodology to supplant or 
add new dimensions to a previous framework; and 
\item solving problems and delivering results that markedly influence the scientific field or society.
\end{itemize}

\newpage

\section*{Factor 4: Contributions, Impact, and Stature}

\vspace{-0.2cm}
\begin{center}
  {\color{red} \footnotesize{up to 14,000 characters, including spaces}}\\
\end{center}

% %Factor 4 is the largest piece to the evaluation, and addresses your
% contributions, the impacts of your research, and your stature as a
% researcher/expert in your field.
%
% When describing your contributions, do not simply state you have
% produced x number of scientific papers and don't list every single
% paper here (use Appendix B and C for these lists). Instead describe
% how your work has contributed to your scientific field and
% society. Use examples from the products listed in the appendices to
% support your description.
%%
% %When describing the impact of your work, include how your work has
% had an impact on scientific and/or societal issues; if your work has
% set new research directions for yourself or others; if you have
% developed new methods, techniques or tools to be used by others; and
% how your work has driven management or policy outcome. You may
% reference papers or presentations and other supporting materials
% included in the appendices. Note that Appendix E offers a place to
% put a wide variety of materials that may speak to your impact and
% stature.
% %
% When describing your stature, provide evidence of stature such as
% requests for expert advice or consultation by other professionals or
% managers; requests to exercise leadership on research teams or
% projects; invitations to serve on advisory boards or lead
% committees, workshops and symposia; invitations to present your work
% or address scientific or professional organizations, recognition by
% other professionals or groups, and honors or awards. Reference
% information contained in appendices as appropriate.
% %
% One way to link your contributions, impact, and stature is to
% describe what you think are your 2-3 most significant contributions
% and explain their importance and impact. This used to be section 19
% in the former format and would be appropriate to include in this
% Factor 4 narrative.
% % 
% %8 page limit

This factor focuses on the researcher's total contributions, impact,
and stature as they bear on the
current research assignment.
It is not restricted to present and immediate past accomplishments 
and achievements.
However, recency of accomplishment is important.
Recent research or similar activity is essential to receiving full
credit. \\
Security regulations, proprietary agreements, or other circumstances may prevent publishing 
research results and make it difficult to evaluate the work based on its impact on the larger 
professional community.
Agencies should develop alternative processes to evaluate the impact
of this work.
In such cases, the work will have to be evaluated by means of the best
possible judgment of its importance and the impact it would have if it
could be published.


\paragraph*{Contributions --} The researcher's contributions reflect the knowledge, skills, and experience the 
incumbent brings to the position.
Professional journal articles are an important product of research results for communicating scientific findings to the broader research community; 
however, they are not the only outlet for communicating information.
Journal articles should be balanced with other forms of communication to ensure broad impact from the results of the 
work.
Indicators of the researcher's contributions may include:
\begin{itemize}
\item research publications (for example, journal articles, monographs, books, reviews, agency 
and customer reports, models, maps, and novel interpretative materials); and 
\item innovations and technology transfer.
\end{itemize}

While the quantity of publications, research contributions, and
professional activities represent one measurement of impact on a
field, do not give undue weight to this metric.
Consider primarily the quality, impact, and relevance of the researcher's contributions on the scientific 
community or field.

\paragraph*{Impact -- } Consider whether the researcher: 
\begin{itemize}
\item has an impact on scientific and/or societal issues; 
\item sets new research directions; 
\item develops new methods, techniques, or tools to be used by other researchers; and 
\item drives management and policy outcomes.
\end{itemize}

\paragraph{Stature -- }Stature is established when the researcher is recognized by the scientific field and/or
society, as indicated by: 
requests for expert advice/consultation by other professionals and managers; 
\begin{itemize}
\item requests to exercise leadership on research teams or projects; 
\item invitations to serve on advisory boards; 
\item requests to organize or chair committees, workshops, or symposia; 
\item invitations to address scientific or professional organizations; 
\item invitations to write synthesis papers; 
\item recognition by professional societies and external groups; or 
\item  honors and awards. 
\end{itemize}

A researcher in one field may move into a related field. Such a move does not change Factor 4 credit if, after a reasonably short period, the researcher will perform research work in the new  field at substantially the same level of competence as before.

\subsubsection*{Factor 4 -- Level A (4 points)}

The researcher defines problems, performs background research, develops and executes a research plan, organizes and evaluates results, and prepares reports of findings.
Work is expected to result in, or has resulted in: 
\begin{itemize}
  \item primary authorship of papers or reports filling narrow gaps in an existing framework of knowledge, to corroborate existing theory, or to report findings of limited scope; or coauthorship of a major paper or report of considerable interest to the scientific field;
  \item providing information and technical support on assigned research projects to collaborators and managers; and
  \item recognition for contributing to the project and communicating results outside the agency. 
\end{itemize}

\subsubsection*{Factor 4 -- Level A (12 points)}

The researcher has demonstrated competence and productivity as evidenced by conducting rigorous research of marked originality, soundness, and value.
Work is expected to result in, or has resulted in: 

\begin{itemize}
  \item primary authorship of publications of considerable interest and value to the field;
  \item conceiving and formulating research ideas supporting or leading to productive studies by others;
  \item products that are significant in solving important scientific problems;
  \item  selection to serve on important committees and review panels of technical groups and professional organizations;
  \item  recognition by the scientific community as a significant contributor to the field of study;
  \item acknowledgement of impact by end users as evidenced by favorable reviews or citation in the work of others;
  \item invitations to make presentations to professional societies and others outside the organization on technical matters and management practices in the area of specialization; and
  \item consultation by users and other researchers who are respected in their fields of study.
\end{itemize}

\subsubsection*{Factor 4 -- Level E (20 points)}

The researcher has made outstanding and significant contributions by conducting research in 
either a broad field or a narrow but very specialized field.
The researcher's accomplishments are of such importance and magnitude
that they move science forward.
Research is of such impact that other researchers must take note of it
to keep abreast of developments in the field. \\


Work at this level includes many of the following:
\begin{itemize}
  \item primary authorship of a number of important papers including seminal or synthesis  publications, some of which have had a major impact on advancing the field or are accepted as authoritative in the field;
  \item contributions to inventions, designs, techniques, models, or theories are regarded as major  advances and open the way for further developments or solving problems of great importance to the professional community, the organization, or the public;
  \item being sought as a consultant by colleagues who are themselves recognized experts in the  field;
  \item recognition by the scientific community as an authority in the field; 
  \item requests from highly-respected colleagues to collaborate with the researcher; 
  \item attracting new researchers to the field;
  \item invitations to address or to assume a leadership role in national professional organizations  and associated committees; and
  \item selection to lead research to solve large and complex problems.
\end{itemize}

In addition, researchers at this level typically perform a variety of
advisory activities based on their scientific reputation and standing
such as:

\begin{itemize}
  \item contributing significantly to professional symposia defining the
  state of the discipline and  new or emerging areas in the field;
  \item contributing to strategic research planning and program development;
  \item participating in major technology or information transfer activities of great importance to the scientific field, the agency, or the public; or
  \item participating in applying the research to important management and policy decisions
\end{itemize}

\newpage

\section*{Three Significant Contributions}

\vspace{-0.2cm}
\begin{center}
  {\color{red} \footnotesize{up to 10,500 characters, including spaces}}\\
\end{center}

\subsection*{Contribution 1}

\noindent \ref{example1}
\textbf{Powell JW}, Smith J.
1881.
Example pub 1.
\textbf{Awesome Journal}. 14:2243-2249.
\url{https://doi.org/10.1111/2041-210X.14156}.


\paragraph{Background--} 

Describe the societal or scientific issue being addressed and, as appropriate, the stakeholders involved. Consider adding a 1-2 sentence statement explaining why you chose to include this contribution.

\paragraph{Role--}

Describe your role in the activity. Be specific.

\paragraph{Results--} 

Describe the findings.

\paragraph{Impact--} 

Describe the scientific and/or societal relevance of these results, how the findings or products been used by others, and what significant changes have been made based on this activity.

\newpage
  
\subsection*{Contribution 2}

\noindent
\ref{example2}: \textbf{Powell JW}, Doe JB, Doe JB.
1882.
Example pub 2.
\textbf{More Awesome Journal}. 14:2243-2249.
\url{https://doi.org/10.1111/2041-210X.14156}.

 
\paragraph{Background--} 

Describe the societal or scientific issue being addressed and, as appropriate, the stakeholders involved. Consider adding a 1-2 sentence statement explaining why you chose to include this contribution.

\paragraph{Role--}

Describe your role in the activity. Be specific.

\paragraph{Results--} 

Describe the findings.

\paragraph{Impact--} 

Describe the scientific and/or societal relevance of these results, how the findings or products been used by others, and what significant changes have been made based on this activity.


\newpage

\subsection*{Contribution 3}

\noindent
\ref{example3}: \textbf{Powell JW}, Doe J, Smith J.
1883.
Example pub 3.
\textbf{Most Awesome Journal}. 14:2243-2249.
\url{https://doi.org/10.1111/2041-210X.14156}.

\paragraph{Background--} 

Describe the societal or scientific issue being addressed and, as appropriate, the stakeholders involved. Consider adding a 1-2 sentence statement explaining why you chose to include this contribution.

\paragraph{Role--}

Describe your role in the activity. Be specific.

\paragraph{Results--} 

Describe the findings.

\paragraph{Impact--} 

Describe the scientific and/or societal relevance of these results, how the findings or products been used by others, and what significant changes have been made based on this activity.


\newpage

\section*{Supporting Information}

\subsection{Current and Recent Projects}

\subsubsection{Exmaple 1} \label{project1}

\begin{itemize}
\item \textit{Role:}
 Principal investigator.
\item \textit{Dates:} 1881.
\item \textit{Funding:} USGS appropriated funding.
\item \textit{Description:} What you did
\end{itemize}

\subsubsection{Exmaple 2} \label{project2}

\begin{itemize}
\item \textit{Role:}
 Principal investigator.
\item \textit{Dates:} 1881.
\item \textit{Funding:} USGS appropriated funding.
\item \textit{Description:} What you did
\end{itemize}

\subsection{Bibliography}

ORCID: \url{https://orcid.org/0000-0003-4649-482X} \\
Google Scholar Profile: \url{https://goo.gl/Tolfoz} \\
ResearchGate Profile: \url{https://www.researchgate.net/profile/Richard_Erickson} \\
Web of Science Profile: \url{https://www.webofscience.com/wos/author/record/AAU-4957-2020} \\

\newcounter{paperCount}

\subsubsection*{Published Products}

\begin{enumerate}
  \item
  \textbf{Powell JW}.
  1876.
  My undergrad paper's title.
  \textbf{The Journal of Awesomeness} 72:575--579.
  \url{https://doi.org/10.2193/2007-161}.\label{paper1}
  
  \item
  \textbf{Powell JW}, Advisor JQ.
  1877.
  My graduate paper 1's title.
  \textbf{The Journal of Awesomeness} 72:575--579.
  \url{https://doi.org/10.2193/2007-161}.\label{paper2}

  \item
  \textbf{Powell JW}, Advisor JQ.
  1877.
  My graduate paper 2's title.
  \textbf{The Journal of Awesomeness} 72:575--579.
  \url{https://doi.org/10.2193/2007-161}.\label{paper3} \\

  \noindent\makebox[\linewidth]{\rule{\textwidth}{1pt}}
  \textbf{Start of USGS service} 

  \item
  \textbf{Powell JW}, Smith J.
  1881.
  Example pub 1.
  \textbf{Awesome Journal}. 14:2243-2249.
  \url{https://doi.org/10.1111/2041-210X.14156}.\\
  IP-1, BAO signed 01/01/1881  \\
  Concept: 50\%; Data: 75\%; Interpretation: 75\%;
  Writing: 76\% \label{example1}
  
  \begin{enumerate}
    \item
    \textbf{Powell JW}, Smith J.
      Data supporting example publication 1.
      U.S.~Geological Survey data release.
      \url{https://doi.org/10.5066/F7KS6PPB} \\
      IP-2, BAO signed 01/01/1881
    \item[\theenumi.SR]
    \textbf{Powell JW}, Smith J.
    Software supporting example publication 1.
    U.S.~Geological Survey software release.
    \url{https://doi.org/10.5066/F7KS6PPB} \\
    IP-3, BAO signed 01/01/1881 
  \end{enumerate}

  
  \noindent\makebox[\linewidth]{\rule{\textwidth}{1pt}} 
  
  \textbf{Review and Promotion: 1885 RGE Panel} 

  \item 
  \textbf{Powell JW}, Doe JB\(^*\), Doe JB.
  1882
  Example pub 2.
  \textbf{More Awesome Journal}. 14:2243-2249.
  \url{https://doi.org/10.1111/2041-210X.14156}.\\
  \(^*\) denotes undergraduate mentored. \\
  IP-1, BAO signed 01/01/1881  \\
  Concept: 50\%; Data: 75\%; Interpretation: 75\%;
  Writing: 76\% \label{example2}
  
  \begin{enumerate}
    \item
    \textbf{Powell JW}, Doe JB, Doe JB.
      Data supporting example publication 2.
      U.S.~Geological Survey data release.
      \url{https://doi.org/10.5066/F7KS6PPB} \\
      IP-2, BAO signed 01/01/1881
    \item[\theenumi.SR]
    \textbf{Powell JW}, Doe JB, Doe JB.
    Software supporting example publication 2.
    U.S.~Geological Survey software release.
    \url{https://doi.org/10.5066/F7KS6PPB} \\
    IP-3, BAO signed 01/01/1881
  \end{enumerate}

  \noindent\makebox[\linewidth]{\rule{\textwidth}{1pt}}

  \textbf{Last review and promotion: 1889 RGE Panel} 
  
  \item 
  \textbf{Powell JW}, Doe J, Smith J\(^\spadesuit\).
  1883
  Example pub 2.
  \textbf{Most Awesome Journal}. 14:2243-2249.
  \url{https://doi.org/10.1111/2041-210X.14156}.\\
  \(^\spadesuit\) denotes postdoc  mentored.\\
  IP-1, BAO signed 01/01/1881  \\
  Concept: 50\%; Data: 75\%; Interpretation: 75\%;
  Writing: 76\% \label{example3}

  \begin{enumerate}
    \item
    \textbf{Powell JW}, Doe JB, Doe JB.
      Data supporting example publication 3.
      U.S.~Geological Survey data release.
      \url{https://doi.org/10.5066/F7KS6PPB} \\
      IP-2, BAO signed 01/01/1881
    \item[\theenumi.SR]
    \textbf{Powell JW}, Doe JB, Doe JB.
    Software supporting example publication 3.
    U.S.~Geological Survey software release.
    \url{https://doi.org/10.5066/F7KS6PPB} \\
    IP-3, BAO signed 01/01/1881 
  \end{enumerate}
  \setcounter{paperCount}{\value{enumi}}
 \end{enumerate}

 \subsubsection*{Products approved for publication}
 
\begin{enumerate}
  \setcounter{enumi}{\value{paperCount}}
  \item example placeholder for numbering
  \setcounter{paperCount}{\value{enumi}}
\end{enumerate}

\subsubsection*{Unpublished technical reports}

\begin{enumerate}
  \setcounter{enumi}{\value{paperCount}}
  \item example placeholder for numbering 
  \setcounter{paperCount}{\value{enumi}}
\end{enumerate}
  
\subsubsection*{Submitted manuscripts}

\begin{enumerate}
  \setcounter{enumi}{\value{paperCount}}
  \item example placeholder for numbering 
  \setcounter{paperCount}{\value{enumi}}
\end{enumerate}

\subsection{Presentations}

\newcounter{presentCount}

\subsubsection*{Invited or noteworthy presentation}

\begin{enumerate}
\item 
\textbf{Powell JW}.
September 1881.
Cool talk title.
Capitol Hill Reception.
Washington, D.C.
\textbf{Invited and Presented}.\label{talk1example}

\noindent\makebox[\linewidth]{\rule{\textwidth}{1pt}}
\textbf{Last review and promotion: 1889 RGE Panel} 

\item Example talk here
\setcounter{presentCount}{\value{enumi}}
  
\end{enumerate}


\subsubsection*{Contributed presentations (since 2017)}

\begin{enumerate}
  \setcounter{enumi}{\value{presentCount}}

  \item
  example regular talk here.
  \item
  example regular talk 3 here.
  
\end{enumerate}

\subsection{Professional and Scientific Service}

\begin{enumerate}


\item Scientific review panels and workshops 
  \begin{itemize}
  \item \textbf{Workshop 1},
    April 2014 \\
    Venue.
    City.\\
    Briefly describe if needed.
  \end{itemize}
  
\item Editorial boards

  \begin{itemize}
  \item \textbf{Editorial Board Member}, Journal 1.  2016--Present
  \item \textbf{Reviewing Editor}, Journal 2.
    2019--2021\label{ERed}
  \item \textbf{Associate Editor}, Journal 3, 2023--present\label{jfwm}
  \end{itemize}

\item Society Service 

  \begin{itemize}
  \item Example committee 1 (2015--present)
  \item Example committee 2 (2010--present)
  \end{itemize}

\item Conference planning

  \begin{itemize}
  \item Assisted in organizing Meeting 2014
  \item Assisted in organizing Meeting 2024
  \end{itemize}
  
\item \textbf{Journal referee.} I typically review 1--3 journal
  articles per month.
  Most recent reviews are summarized on my Web of Science (previously
  Publons) Profile
  (\url{https://www.webofscience.com/wos/author/record/1122397}).
  I have reviewed for the following journals (listed
  alphabetically):\\
  \textit{A journal},
  \textit{B journal},
  and the \textit{C journal}.
\end{enumerate}

\subsection{Academic Service}\label{acSer}

\subsubsection*{Academic appointments}

\begin{itemize}
\item Adjunct Professor (Courtesy Appointment, no rank),
University of State, Department of Awesomeness (2014-Present)
\end{itemize}

\subsection*{Students or postdocs advised or mentored}\label{mentor}

\begin{itemize}
\item Jane Done.
  MS graduate committee member.
  State University.
  Fall 2024--present.
\item Dr.~John Doe.
  Postdoc.
  Co-mentored with Jane Smith.
  University of State.
  Fall 2023--Present.
\item Sarah Student,
  Research Intern.
  State School.
  June 2021--September 2022.
  Next position: Acme Inc.
\end{itemize}
 
\subsubsection*{Courses taught and seminars presented}

\begin{enumerate}
\item A course at National Meeting. Location. January 2007.
\item Cool research talk.
Weekly seminar.
Department of Awesomeness.
State University.
April 2008.
 
  \noindent\makebox[\linewidth]{\rule{\textwidth}{1pt}} 
  \textbf{Start of USGS service}
  
\item Example talk.
    
  \noindent\makebox[\linewidth]{\rule{\textwidth}{1pt}}
  \textbf{Review and promotion: 2017 RGE Panel} 
  
\item Another workshop. 

  \noindent\makebox[\linewidth]{\rule{\textwidth}{1pt}}
  \textbf{Last review and promotion: 2021 RGE Panel} 

\item newer workshop

\item Guest lecture.
\end{enumerate}

\subsection{Technical Training Provided}\label{TTP}

\begin{enumerate}
\item Smith JD and \textbf{Powell JW}. Training course of rafting. North American Rafting Meeting. Colorado River, CO. Full-day course. November 1881.

  \noindent\makebox[\linewidth]{\rule{\textwidth}{1pt}} 
  \textbf{Start of USGS service}

\item example item

  \noindent\makebox[\linewidth]{\rule{\textwidth}{1pt}} 
  \textbf{Review and promotion: 2017 RGE Panel} 

\item example item.
  \noindent\makebox[\linewidth]{\rule{\textwidth}{1pt}} 
  \textbf{Last review: 2021 RGE Panel} 

\item example 
   
\end{enumerate}

\subsection{Other mentorship}\label{mentor:other}

\begin{itemize}
\item Dr.~Jane Doe,
  Faculty on sabbatical learning about occupancy modeling and applied
  Bayesian statistics.
  Also, I assisted Dr.~Doe is using USGS data for teaching and
  education outreaching including high-school students, college
  students, graduate students, and continuing education for middle and
  high school teachers.
  State school.
  Fall 2019.
 
\end{itemize}



\subsection{Awards and Recognition}

\begin{itemize}
\item Award 1.
\item Award 2.
  
\end{itemize}


\subsection{Special Assignments}

None.

\subsection{Inventions and Patents}

None.

\subsection{Outreach and Media Coverage}

\begin{enumerate}
\item Outreach
  \begin{itemize}
  \item Example 1.
  \end{itemize}
  
\item Media coverage\label{sec:TITD}

  \begin{itemize}
  \item Example 1
  \end{itemize}
\end{enumerate}

\subsection{Previous Professional Positions}

Overview if needed.
Otherwise, list positions.

\begin{itemize}
\item \textit{Geologist (GS-0482-11) \hfill January 1877--April 1878}
\item \textit{Adventurer (GS-0401-11)  	 \hfill June 1878--January
    1881} 
\end{itemize}


\subsection{Education}

\begin{itemize}
\item University, \textit{Doctorate of Philosophy}, August 2013\\
Major: Rocks, Minor: Botany
\item  University, \textit{Masters of Science}, August 2009\\
Major: Rocks
\item University of State, \textit{Bachelors of Science}, May 2007\\
Majors:Rocks and Water, Minor: Fish
\end{itemize}


\subsubsection*{\small{Privacy Act Notice:}}
\vspace{-0.3cm}
\begin{adjustwidth}{0cm}{}
  {\footnotesize
Pursuant to Section 3(e)(3) of the Privacy Act of 1974 (Public Law
93-573), the individual furnishing information on this form is hereby
advised as follows: 1. The authority for solicitation of the
information is 5 USC 552(a). 2. The principal purpose for which the
information is intended to be used is for the U.S. Geological Survey
research and development peer panel review process. 3. The routine
disclosure of the information is to scientific, management and
administrative staff who are participants in the peer review process
or who are in the human resources office. 4. The effect on the
individual of not providing all or any part of the requested
information is not having an up-to-date Research and Development
Scientific Record for peer review thereby resulting in a delayed or no
peer review. 5. This record and information in this record may be used
by the Federal government in connection with the hiring of an
employee, the issuance of a security clearance, the conducting of a
security or suitability investigation of an individual, the
classifying of jobs, the letting of a contract, and the issuance of a
license, grant, or other benefits or awards to the extent that the
information is relevant and necessary.
}
\end{adjustwidth}


\newpage

\section*{REFERENCES}

\begin{multicols}{2}

  \begin{itemize}
  \item[] Mr.~Fancy Pants\\
    Center Director\\
    555.123.4567 \\
    fpants@usgs.gov \\
    {\color{gray} Relationship to you:} Center Director

    \item[] Dr.~I.B.~Important\\
    Supervisory Supervisor\\
    555.123.4567 \\
    ibimportrant@usgs.gov \\
    {\color{gray} Relationship to you:} Supervisor/Branch Chief

  \item[] Ms.~Their Name.\\
    Title
    555.123.4567 \\
    name@state.gov  \\
    {\color{gray} Relationship to you:} Program
    coordinator/collaborator. Describe impact of research.

    \item[] Ms.~Their Name.\\
    Title
    555.123.4567 \\
    name@state.gov  \\
    {\color{gray} Relationship to you:} Program
    coordinator/collaborator. Describe impact of research.
       % \columnbreak
 
  \end{itemize}

  
\end{multicols}



\end{document}
