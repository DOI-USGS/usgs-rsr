% %Factor 4 is the largest piece to the evaluation, and addresses your
% contributions, the impacts of your research, and your stature as a
% researcher/expert in your field.
%
% When describing your contributions, do not simply state you have
% produced x number of scientific papers and don't list every single
% paper here (use Appendix B and C for these lists). Instead describe
% how your work has contributed to your scientific field and
% society. Use examples from the products listed in the appendices to
% support your description.
%%
% %When describing the impact of your work, include how your work has
% had an impact on scientific and/or societal issues; if your work has
% set new research directions for yourself or others; if you have
% developed new methods, techniques or tools to be used by others; and
% how your work has driven management or policy outcome. You may
% reference papers or presentations and other supporting materials
% included in the appendices. Note that Appendix E offers a place to
% put a wide variety of materials that may speak to your impact and
% stature.
% %
% When describing your stature, provide evidence of stature such as
% requests for expert advice or consultation by other professionals or
% managers; requests to exercise leadership on research teams or
% projects; invitations to serve on advisory boards or lead
% committees, workshops and symposia; invitations to present your work
% or address scientific or professional organizations, recognition by
% other professionals or groups, and honors or awards. Reference
% information contained in appendices as appropriate.
% %
% One way to link your contributions, impact, and stature is to
% describe what you think are your 2-3 most significant contributions
% and explain their importance and impact. This used to be section 19
% in the former format and would be appropriate to include in this
% Factor 4 narrative.
% % 
% %8 page limit

This factor focuses on the researcher's total contributions, impact,
and stature as they bear on the
current research assignment.
It is not restricted to present and immediate past accomplishments 
and achievements.
However, recency of accomplishment is important.
Recent research or similar activity is essential to receiving full
credit. \\
Security regulations, proprietary agreements, or other circumstances may prevent publishing 
research results and make it difficult to evaluate the work based on its impact on the larger 
professional community.
Agencies should develop alternative processes to evaluate the impact
of this work.
In such cases, the work will have to be evaluated by means of the best
possible judgment of its importance and the impact it would have if it
could be published.


\paragraph*{Contributions --} The researcher's contributions reflect the knowledge, skills, and experience the 
incumbent brings to the position.
Professional journal articles are an important product of research results for communicating scientific findings to the broader research community; 
however, they are not the only outlet for communicating information.
Journal articles should be balanced with other forms of communication to ensure broad impact from the results of the 
work.
Indicators of the researcher's contributions may include:
\begin{itemize}
\item research publications (for example, journal articles, monographs, books, reviews, agency 
and customer reports, models, maps, and novel interpretative materials); and 
\item innovations and technology transfer.
\end{itemize}

While the quantity of publications, research contributions, and
professional activities represent one measurement of impact on a
field, do not give undue weight to this metric.
Consider primarily the quality, impact, and relevance of the researcher's contributions on the scientific 
community or field.

\paragraph*{Impact -- } Consider whether the researcher: 
\begin{itemize}
\item has an impact on scientific and/or societal issues; 
\item sets new research directions; 
\item develops new methods, techniques, or tools to be used by other researchers; and 
\item drives management and policy outcomes.
\end{itemize}

\paragraph{Stature -- }Stature is established when the researcher is recognized by the scientific field and/or
society, as indicated by: 
requests for expert advice/consultation by other professionals and managers; 
\begin{itemize}
\item requests to exercise leadership on research teams or projects; 
\item invitations to serve on advisory boards; 
\item requests to organize or chair committees, workshops, or symposia; 
\item invitations to address scientific or professional organizations; 
\item invitations to write synthesis papers; 
\item recognition by professional societies and external groups; or 
\item  honors and awards. 
\end{itemize}

A researcher in one field may move into a related field. Such a move does not change Factor 4 credit if, after a reasonably short period, the researcher will perform research work in the new  field at substantially the same level of competence as before.

\subsubsection*{Factor 4 -- Level A (4 points)}

The researcher defines problems, performs background research, develops and executes a research plan, organizes and evaluates results, and prepares reports of findings.
Work is expected to result in, or has resulted in: 
\begin{itemize}
  \item primary authorship of papers or reports filling narrow gaps in an existing framework of knowledge, to corroborate existing theory, or to report findings of limited scope; or coauthorship of a major paper or report of considerable interest to the scientific field;
  \item providing information and technical support on assigned research projects to collaborators and managers; and
  \item recognition for contributing to the project and communicating results outside the agency. 
\end{itemize}

\subsubsection*{Factor 4 -- Level A (12 points)}

The researcher has demonstrated competence and productivity as evidenced by conducting rigorous research of marked originality, soundness, and value.
Work is expected to result in, or has resulted in: 

\begin{itemize}
  \item primary authorship of publications of considerable interest and value to the field;
  \item conceiving and formulating research ideas supporting or leading to productive studies by others;
  \item products that are significant in solving important scientific problems;
  \item  selection to serve on important committees and review panels of technical groups and professional organizations;
  \item  recognition by the scientific community as a significant contributor to the field of study;
  \item acknowledgement of impact by end users as evidenced by favorable reviews or citation in the work of others;
  \item invitations to make presentations to professional societies and others outside the organization on technical matters and management practices in the area of specialization; and
  \item consultation by users and other researchers who are respected in their fields of study.
\end{itemize}

\subsubsection*{Factor 4 -- Level E (20 points)}

The researcher has made outstanding and significant contributions by conducting research in 
either a broad field or a narrow but very specialized field.
The researcher's accomplishments are of such importance and magnitude
that they move science forward.
Research is of such impact that other researchers must take note of it
to keep abreast of developments in the field. \\


Work at this level includes many of the following:
\begin{itemize}
  \item primary authorship of a number of important papers including seminal or synthesis  publications, some of which have had a major impact on advancing the field or are accepted as authoritative in the field;
  \item contributions to inventions, designs, techniques, models, or theories are regarded as major  advances and open the way for further developments or solving problems of great importance to the professional community, the organization, or the public;
  \item being sought as a consultant by colleagues who are themselves recognized experts in the  field;
  \item recognition by the scientific community as an authority in the field; 
  \item requests from highly-respected colleagues to collaborate with the researcher; 
  \item attracting new researchers to the field;
  \item invitations to address or to assume a leadership role in national professional organizations  and associated committees; and
  \item selection to lead research to solve large and complex problems.
\end{itemize}

In addition, researchers at this level typically perform a variety of
advisory activities based on their scientific reputation and standing
such as:

\begin{itemize}
  \item contributing significantly to professional symposia defining the
  state of the discipline and  new or emerging areas in the field;
  \item contributing to strategic research planning and program development;
  \item participating in major technology or information transfer activities of great importance to the scientific field, the agency, or the public; or
  \item participating in applying the research to important management and policy decisions
\end{itemize}
