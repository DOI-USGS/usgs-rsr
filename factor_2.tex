%This factor addresses your level of independence. Do you work on
%projects as assigned under close supervision, which is where most
%everyone starts, or do you define and pursue your own research
%projects  with little or no supervisory controls?
%
%Describe your relationship with your supervisor and your current level of independent performance.  Specify the following:
%?	what is your supervisor?s role (e.g., Science Center Director, supervisory scientist, ..);
%?	how work is assigned (e.g., consider nature of guidance provided, employee?s freedom in choosing a course of action);
%?	how independently work is performed;
%?	technical and administrative guidance and control exercised by supervisor
%?	how work is reviewed and the acceptance of your research results, conclusions and final products; and
%?	freedom to represent your research program to other organizations and agencies, policy makers and the public. 
%
%Use caution in distinguishing between consultation and supervisory control and guidance.
%1-page limit

This factor deals with the researcher’s current level of independent performance and the technical and administrative guidance and control the supervisor exercises over research work. 
Researchers may consult frequently with colleagues and collaborators.
Use caution in distinguishing between consultation and supervisory control and guidance

\textbf{Primary Considerations} -- In evaluating this factor, consider
the following:

\begin{itemize}
\item manner in which the supervisor assigns work;
\item researcher's freedom to determine a course of action; 
\item researcher's opportunity for procedural innovation; and 
\item degree of the supervisor's acceptance of the researcher's recommendations, decisions, and 
  final products.
\end{itemize}
Researchers working on complex team projects not divided into smaller components exercise 
independent performance when they:
\begin{itemize}
\item participate fully as a professionally responsible team member in substantive aspects of the 
  work; and 
\item make contributions equivalent to independently performing more
  limited research projects.
\end{itemize}

\subsubsection*{Factor 2 – Level A (2 points)}

The supervisor typically assigns specific problems along with general instructions on the scope and objectives of the study.
The supervisor or higher management makes any decisions to discontinue work, change emphasis, or change the research plan.
The researcher may suggest  studies and undertake them after receiving supervisory approval.
The supervisor reviews work for adequacy of method, completeness, and appropriate interpretation of results. \\

The researcher confers with the supervisor regarding problem definition, the relationship of the problem to the organization’s broader research goals, and developing a research plan. 
Supervisory or managerial direction and guidance help the researcher in the critical problem definition and planning stages, but do not negate the researcher’s responsibility for adequately completing these steps.  \\

The researcher is expected to:

\begin{itemize}
  \item assume responsibility for the study and pursue it to completion;
  \item solve problems ordinarily encountered in accomplishing the work with only occasional supervisory input;
  \item interpret results; and
  \item prepare entire, or sections of, reports and papers.
\end{itemize}

\subsubsection*{Factor 2 -- Level C (6 points)}

The supervisor may either assign a broad problem area to the researcher or allow the researcher to work with substantial freedom within an area of primary interest.
The researcher has substantial freedom to identify, define, and select specific projects, and to determine the most promising research strategies and problem approaches. \\

The supervisor:
\begin{itemize}
  \item approves plans calling for considerable investments of time or resources;
  \item makes final decisions concerning the direction of work and changes in or discontinuance of projects involving substantial research investments;
  \item relies on the researcher's professional judgment to such an extent that the researcher's recommendations are ordinarily followed; and
  \item  reviews final work and reports, principally to evaluate overall results, recommendations, and conclusions.
\end{itemize}

The researcher is responsible, with little technical direction, for: 
\begin{itemize}
  \item formulating hypotheses;
  \item developing and carrying out the research plan;
  \item determining equipment and other resource needs;
  \item keeping the supervisor informed of general plans and progress;
  \item addressing novel and difficult problems requiring modification of standard methods;
  \item analyzing and interpreting results;
  \item preparing comprehensive reports of findings; and
  \item working with users to interpret and implement research findings or technologies,
\end{itemize}


\subsubsection*{Factor 2 -- Level E (10 points)}

The supervisor provides broad administrative supervision, which is
generally limited to approving staffing, funds, and facilities, and
to providing broad guidance on agency policies and mandates.
Technical supervision is consultative in nature.
Management accepts the researcher's findings as technically
authoritative, as a basis for decisions, and as acceptable for review
by the scientific community.
The researcher, working within the framework of management objectives
and priorities, is responsible for:
\begin{itemize}
  \item formulating research plans and hypotheses; 
  \item carrying out the project plan; 
  \item interpreting findings and assessing their organizational and professional applicability; and 
  \item locating and exploring the most promising areas of research in relation to agency program needs and the state of the science or discipline.
\end{itemize}