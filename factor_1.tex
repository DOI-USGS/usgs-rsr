%Describe your research assignment and define the primary areas(s) of work, with attention to:
%?	scope and breadth of the research;
%?	availability of related research studies or products, or protocols;
%?	complexity of assignment (e.g., unknowns; critical obstacles; difficulty)
%?	extent to which objectives can be defined;
%?	approach(s)taken to solve scientific or engineering problems;
%?	variety and depth of knowledge and expertise required to solve problems;
%?	extent and complexity of the required validation process;
%?	necessity to translate abstract concepts into easily understood statements of theory or models;
%?	importance of and utility of end products in solving problems and for use by managers, decision and policy makers; and
%?	importance of work in opening new areas of research or development.
%
%Do not describe each individual study or project but rather focus on the major themes of your research program. Individual study/project titles and funding sources can be listed in Appendix A and can be referenced in the narrative. 
%%2-page limit

%?	scope and breadth of the research;

This factor deals with the nature, scope, and characteristics of the researcher's current assignment.
Award a factor level that reflects the norm of current assignments,
rather than atypical projects.
Research assignments are directly dependent upon the individual
qualities of the researcher and the inherent difficulty of the
research problems.
Work commonly expands commensurate with the researcher's motivation,
capability, and creativity. \\

\paragraph*{\textbf{Projects and Teams -- }}  For project and team members, base the factor
level only on the specific projects or portion of projects for which
the researcher is responsible.
For project managers, base the factor level on the scope and character
of the total project.

\paragraph*{\textbf{Primary Considerations --}}
 In evaluating this factor consider the
following:
\begin{itemize}
  \item assignment scope and complexity, objectives, and means of accomplishment;
  \item problem breadth and depth; 
  \item availability of related research studies; 
  \item extent to which objectives can be defined; 
  \item number of unknowns and critical obstacles; 
  \item variety and depth of knowledge and expertise required to solve problems;
  \item extent and complexity of the required validation process; 
  \item necessity to translate abstract concepts into easily understood statements of theory or models, and to determine how best to disseminate information or transfer research findings;
  \item utility of the end product in solving the initial problem and in opening new areas of investigation; and
  \item expected impact of end results, products, or outcomes.
\end{itemize}

\subsubsection*{Factor 1 --  Level A (2 points)}

Research assignments have the following characteristics:
\begin{itemize}
  \item readily definable objectives; 
  \item limited in scope to investigating specific phenomena or problems, or are segments of related investigations; 
  \item require fairly conventional techniques; 
  \item involve applying existing theory or methods to areas previously investigated, but under different conditions, or involve adapting previous studies in light of changes in theory or improved techniques and instrumentation; and
  \item result in contributions that add to scientific and professional knowledge or support developing new or improved methods and techniques. 
\end{itemize}

\noindent The researcher typically works as a project or team member. 

\subsubsection*{Factor 1 --  Level C (6 points)}

Research assignments have the following characteristics:
\begin{itemize}
  \item the scope is broad and complex, requiring a series of comprehensive and conceptually related phases and studies; 
  \item problems are difficult to define;
  \item require sophisticated research techniques; and
  \item result in contributions that:
  \begin{itemize}
    \item answer important questions in the field;
    \item account for previously unexplained phenomena; 
    \item open significant new avenues for further study;
    \item confirm or modify a scientific theory or methodology;
    \item lead to important changes in existing products, methods, techniques, processes, or practices; or
    \item are definitive of a specific topic area. 
  \end{itemize}
\end{itemize}

\noindent The researcher typically works as a project member or as a primary investigator. 
\subsubsection*{Factor 1 --  Level E (10 points)}

Research assignments have the following characteristics:
\begin{itemize}
  \item  the scope and complexity are at a level requiring subdivision into separate phases, some of  which are considerably broad and complex; 
  \item problems are exceptionally difficult and unyielding to investigation; 
  \item require unconventional or novel approaches or complex research techniques; and 
  \item results may include:
  \begin{itemize}
    \item a major advance or opening of the way for extensive related development; 
    \item progress in areas of exceptional interest to the scientific and professional community; 
    \item important changes in theories, methods, and techniques; 
    \item opening significant new avenues for further study; or 
    \item contributions answering important questions in the field.
  \end{itemize}
\end{itemize} 

\noindent The researcher typically works as a primary investigator but may also be a project member.

